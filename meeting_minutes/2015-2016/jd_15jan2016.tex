% Copyright (c) 2015 IEEE Student Branch of Granada - All Rights Reserved.
% This work is licensed under the Creative Commons Attribution 4.0
% International License. To view a copy of this license, visit
% http://creativecommons.org/licenses/by/4.0/.

% Template
\documentclass[12pt,twoside,openany,a4paper]{book}
\pdfoutput=1

% Margins
\usepackage[a4paper,top=1cm,bottom=6cm]{geometry}

% Paragraphs
\usepackage{parskip}            % For paragraph separation
\setlength{\parindent}{15pt}    % Paragraph indentation
\widowpenalties=1 10000         % No page break in inside pragraph
\raggedbottom{}

% Font
\usepackage[T1]{fontenc}        % Output font
\usepackage[utf8]{inputenc}     % Input encoding
\usepackage[english,spanish]{babel} % For Spanish texts
\usepackage{FiraSans}           % Use a better font

% Extra packages
\usepackage{graphicx}           % For graphics
\usepackage{verbatim}           % For non-parsed text blocks
\usepackage{microtype}          % Improve PDF text lines
\usepackage{indentfirst}        % Indent first paragraph too
\usepackage{hyperref}           % For intenal and external links
\usepackage[hypcap]{caption}    % For captions
\usepackage{ctable}             % For tables
\usepackage{float}              % For table positions
\usepackage{subcaption}         % For subfigures
\usepackage{fancyhdr}           % For headers and footers
\usepackage{xcolor}             % Colors everywhere
\usepackage{titlesec}           % For customize the title format

% Document properties
\makeatletter
\author{Benito Palacios Sánchez}  \let\Author\@author
\title{Junta Directiva}           \let\Title\@title
\date{15 de enero de 2016}        \let\Date\@date
\makeatother

\hypersetup{
    pdfauthor = {\Author},
    pdftitle  = {\Title},
    pdfsubject  = {},
    pdfkeywords = {meeting, ieee},
    pdfcreator  = {TeXLive distro and Atom text editor},
    pdfproducer = {pdflatex}
}

% Define custom colors
\definecolor{BlueTitle}{HTML}{2E5395}

% Define the style for header and footer
\newcommand{\fncymain}{%
    \setlength{\headheight}{85pt}
    \fancyhead[RO,LO,RE,LE]{}
    \fancyhead[C]{
        \includegraphics[height=0.10\textheight,keepaspectratio]
            {../logo_header.png}}
    \renewcommand{\headrulewidth}{0pt}

    \fancyfoot[C]{
        \centering
        \textbf{IEEE Student Branch of Granada} \\
        Campus Fuentenueva. Facultad de Ciencias. Severo Ochoa, S/N. \\
        Sala de alumnos. 18001, Granada, España. \\
        E-mail: \href{mailto:ieeegranada@ieee.org}{ieeegranada@ieee.org}  \\
        Web: \url{http://ieee-urg.org}
    }
}

% Create the document title
\renewcommand{\maketitle}{
    {
        \Huge
        \color{BlueTitle}
        \centering
        \bfseries
        \scshape
        \hrulefill \\
        \Title \\
        \hrulefill
    }
}

% Set the section format
\titleformat{\section}[display]{
    \LARGE
    \color{BlueTitle}
    \filcenter
    \bfseries
    \scshape
    }{}{0pt}{}[{\titlerule[1pt]}]

% Set only one number for subsections
\renewcommand{\thesubsection}{\arabic{subsection}}

% Start!
\begin{document}
    % Page style
    \pagestyle{fancy}
    \fncymain{}

    % Add the title
    \maketitle

    % Date, participantes and location info.
    En Granada, siendo las 8:35 horas del \Date~se encuentran reunidos:
    \begin{itemize}
        \item Benito Palacios Sánchez
        \item Sulaimane Mezzouji
        \item Julián Fernandez Ortiz
        \item Laura Hinojosa Vico
        \item Javier Hidalgo Atienza
        \item Pilar López Varo
    \end{itemize}

    \section{Orden del día}
    \begin{enumerate}
        \item Actividades en desarrollo.
        \begin{enumerate}
            \item Conferencia de creatividad artificial y aplicación a la medicina
            \item Wall-E
        \end{enumerate}
        \item Traspaso de tesorería
        \item Compra de polos
        \item Página Web
    \end{enumerate}


    \section{Desarrollo de la sesión}
    \subsection{Actividades en desarrollo}

    \subsubsection{Conferencia de creatividad artificial y aplicación a la medicina}
    Julián ha estado realizando la cooperación con la asociación UNIGRAMA para el desarrollo de esta charla. Se reunió con el presidente de dicha asociación y se repartieron las tareas de difusión. Toda la cartelería a cargo de nuesta asociación está realizada e impresa. Además se ha enviado correos a las listas de difusion y los carteles se han expuesto en los monitores de la ETSIIT. Se ha puesto en contacto para la difusión del correo en las listas de correo de toda la universidad con el vicerrectorado de estudiantes. La charla trata sobre inteligencia artificial para la práctica de musicología que una empresa está desarrollando.

    \subsubsection{Wall-E}
    Diseño de las piezas: Se están imprimiendo las partes del brazo. Seguramente será necesario una remodelación del diseño tras realizar varias pruebas. Luego faltaría crear documentación.

    Diseño del circuito: No ha habido ninguna contribución todavía. Se le ha avisado que si no contribuyen se tendrá que expulsar y dejar paso a otros miembros que mostraron interés en el proyecto.

    Movimiento: Hay poca información sobre los encoders. Se han encontrado con el problema de los motores se mueven a diferente velocidad con los mismos parametros. El control de ultrasonidos es sencillo y esta practicamente hecho. Se esta esperando a que uno de los miembros realicen documentacion sobre la bateria y la placa, todavia no se ha recibido respuesta.

    El objetivo es que para el mes de febrero la parte de movimiento del coche este en movimiento.

    \subsection{Traspaso de tesorería}
    Una vez termine la junta directiva, tendran una reunion para terminar el traspaso de conocimiento de tesoreria.

    \subsection{Compra de polos}
    Se está esperando a que todos los miembros realicen el pago.

    \subsection{Página Web}
    Hay que actualizar el logo de los patrocinadores de la asociación y los enlaces hacia las correspondientes webs. Además, añadir un enlace personal en los miembros de la junta directiva.

    \subsection{Let's Eat}
    David Téllez ha propuesto el desarrollo e implementación de un dispositivo de cuchara que facilite poder comer de forma autónoma. Su duración será de 9 meses con un presupuesto de menos de 80 euros. El número de participantes será de 2 a 4 personas.

    Se acepta por unanimidad el desarrollo del proyecto.

    Los pasos a seguir es realizar una descripción, difusión y buscar miembros para el mismo.

    \subsection{Simulación de Sistemas de Crecimiento DLA}
    Cristina ha propuesto la realización de un taller sobre el crecimiento DLA. En él se expodrá la idea y se realizarán simulaciones.

    Se acepta por unanimidad el desarrollo del proyecto.

    Los siguientes pasos a seguir es una descripción, difusión y buscar un sitio para su realización. Para socios de IEEE será gratuito. Para el resto se aprueba por mayoría absoluta pagar 3 euros.

    \section{Ruegos, sugerencias y preguntas}
    Ninguna.


    \clearpage
    Siendo las 09:40 horas del mismo día se da por terminada la sesión.

    A \Date.
    \vspace{10mm}
    \ctable[pos = h, width = 140 mm]{XX}{}{
        El Secretario & VB del Presidente \NN
                      &                   \NN
                      &                   \NN
                      &                   \NN
                      &                   \NN
                      &                   \NN
                      &                   \NN
        Fdo.: \Author & Fdo.: Sulaimane Mezzouji
    }
\end{document}
