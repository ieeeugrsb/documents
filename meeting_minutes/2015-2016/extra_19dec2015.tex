% Copyright (c) 2015 IEEE Student Branch of Granada - All Rights Reserved.
% This work is licensed under the Creative Commons Attribution 4.0
% International License. To view a copy of this license, visit
% http://creativecommons.org/licenses/by/4.0/.

% Template
\documentclass[12pt,twoside,openany,a4paper]{book}
\pdfoutput=1

% Margins
\usepackage[a4paper,top=1cm,bottom=6cm]{geometry}

% Paragraphs
\usepackage{parskip}            % For paragraph separation
\setlength{\parindent}{15pt}    % Paragraph indentation
\widowpenalties=1 10000         % No page break in inside pragraph
\raggedbottom{}

% Font
\usepackage[T1]{fontenc}        % Output font
\usepackage[utf8]{inputenc}     % Input encoding
\usepackage[english,spanish]{babel} % For Spanish texts
\usepackage{FiraSans}           % Use a better font

% Extra packages
\usepackage{graphicx}           % For graphics
\usepackage{verbatim}           % For non-parsed text blocks
\usepackage{microtype}          % Improve PDF text lines
\usepackage{indentfirst}        % Indent first paragraph too
\usepackage{hyperref}           % For intenal and external links
\usepackage[hypcap]{caption}    % For captions
\usepackage{ctable}             % For tables
\usepackage{float}              % For table positions
\usepackage{subcaption}         % For subfigures
\usepackage{fancyhdr}           % For headers and footers
\usepackage{xcolor}             % Colors everywhere
\usepackage{titlesec}           % For customize the title format

% Document properties
\makeatletter
\author{Benito Palacios Sánchez}     \let\Author\@author
\title{Asamblea Extradinaria}        \let\Title\@title
\date{19 de diciembre de 2015}         \let\Date\@date
\makeatother

\hypersetup{
    pdfauthor = {\Author},
    pdftitle  = {\Title},
    pdfsubject  = {},
    pdfkeywords = {meeting, ieee},
    pdfcreator  = {TeXLive distro and Atom text editor},
    pdfproducer = {pdflatex}
}

% Define custom colors
\definecolor{BlueTitle}{HTML}{2E5395}

% Define the style for header and footer
\newcommand{\fncymain}{%
    \setlength{\headheight}{85pt}
    \fancyhead[RO,LO,RE,LE]{}
    \fancyhead[C]{
        \includegraphics[height=0.10\textheight,keepaspectratio]
            {../logo_header.png}}
    \renewcommand{\headrulewidth}{0pt}

    \fancyfoot[C]{
        \centering
        \textbf{IEEE Student Branch of Granada} \\
        Campus Fuentenueva. Facultad de Ciencias. Severo Ochoa, S/N. \\
        Sala de alumnos. 18001, Granada, España. \\
        E-mail: \href{mailto:ieeegranada@ieee.org}{ieeegranada@ieee.org}  \\
        Web: \url{http://ieee-urg.org}
    }
}

% Create the document title
\renewcommand{\maketitle}{
    {
        \Huge
        \color{BlueTitle}
        \centering
        \bfseries
        \scshape
        \hrulefill \\
        \Title \\
        \hrulefill
    }
}

% Set the section format
\titleformat{\section}[display]{
    \LARGE
    \color{BlueTitle}
    \filcenter
    \bfseries
    \scshape
    }{}{0pt}{}[{\titlerule[1pt]}]

% Set only one number for subsections
\renewcommand{\thesubsection}{\arabic{subsection}}

% Start!
\begin{document}
    % Page style
    \pagestyle{fancy}
    \fncymain{}

    % Add the title
    \maketitle

    % Date, participantes and location info.
    En Granada, siendo las 11:07 horas del \Date~se encuentran reunidos: 12 socios.


    \section{Orden del día}
    \begin{enumerate}
        \item Nueva pagina web de la rama.
        \item Tipos de socio.
        \item Cambios en los estatutos.
        \item Elecciones.
        \item Análisis general de la rama.
    \end{enumerate}


    \section{Desarrollo de la sesión}
    \subsection{Aprobacion del acta anterior}
    Se aprueba por unanimidad el acta de la sesión anterior.

    \subsection{Nueva pagina web.}
    Sulaimane presenta la nueva pagina web y explica que la zona miembros será un espacio privado al que se accederá mediante una contraseña comunicada a cada socio previamente.

    \subsection{Tipos de socio}
    El presidente propone los siguientes tipos de socios, no excluyentes.
    \begin{itemize}
        \item Socios fundadores: socios que fundaron la rama en sus orígenes.

        \item Socios de números: El estatus se aprueba por la junta directiva.
            \begin{itemize}
                \item Socios de número voluntarios: personas que no siendo socios de IEEE, la junta directiva aprueba que puedan formar parte de la rama y sus actividades, siempre y cuando no entren en conflictos por los requisitos establecidos por IEEE en aquellas actividades exclusivas.

                \item Socios de número de IEEE: personas socios de IEEE que pagan la cuota anualmente. Se incluye \textit{Undergraduate Student Member}, \textit{Graduate Student Member} y \textit{Young Professionals}.
            \end{itemize}

        \item Socios de honor: se determinará por asamblea general.
    \end{itemize}

    Habrá precios diferentes para los socios. Dado que los socios de IEEE se exige que sea gratis, se propone que los socios de número su cuota sea ninguna siempre.

    Se aprueba por unanimidad los nuevos tipos de socios.

    \subsection{Cambios en los estatutos}
    Se aprueba por unanimidad el cambio de género a uno neutro.

    Se aprueba por unanimidad el cambio del periodo de antelación con la que se da a conocer una asamblea.
    Se hace mención también del cambio de domicilio aprobada en la asamblea extraordinara de septiembre.

    \subsection{Elecciones}
    Se da a conocer la dimisión del actual tesorero Julian Fernandez Ortiz. Se aprueba por unanimidad el puesto de tesorería a: Javier Hidalgo Atienza.

    Se aprueba por unanimidad los puestos de vocalía a: Jesús Rodriguez Venzal, Pilar López Varo y Laura Hinojosa Vico.

    \subsection{Analisis general de la rama}
    Comienza una mesa redonda sobre el estado general de la asociacion.

    Sulaimane exponse las actividades que se realizaorn para la hora del codigo con la asociacion Granada Acoge.

    Pilar propone dar cursos de programación y electrónica a los de primero de los grados.

    Laura hace enfásis en dar visibilidad a la rama y comunicar cualquier evento organizado y en el que se participe.

    Benito plantea el uso de una plataforma educativa como edX para impartir cursos gratuitos on-line. Así como da a conocer su intención de importir unas jornadas sobre ingeniería inversa aplicada a videojuegos.

    \clearpage
    Siendo las 12:49 horas del mismo día se da por terminada la sesión.

    A \Date.
    \vspace{10mm}
    \ctable[pos = h, width = 140 mm]{XX}{}{
        El Secretario & VB del Presidente \NN
                      &                   \NN
                      &                   \NN
                      &                   \NN
                      &                   \NN
                      &                   \NN
                      &                   \NN
        Fdo.: \Author & Fdo.: Sulaimane Mezzouji
    }
\end{document}
