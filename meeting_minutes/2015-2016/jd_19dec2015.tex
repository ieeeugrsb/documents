% Copyright (c) 2015 IEEE Student Branch of Granada - All Rights Reserved.
% This work is licensed under the Creative Commons Attribution 4.0
% International License. To view a copy of this license, visit
% http://creativecommons.org/licenses/by/4.0/.

% Template
\documentclass[12pt,twoside,openany,a4paper]{book}
\pdfoutput=1

% Margins
\usepackage[a4paper,top=1cm,bottom=6cm]{geometry}

% Paragraphs
\usepackage{parskip}            % For paragraph separation
\setlength{\parindent}{15pt}    % Paragraph indentation
\widowpenalties=1 10000         % No page break in inside pragraph
\raggedbottom{}

% Font
\usepackage[T1]{fontenc}        % Output font
\usepackage[utf8]{inputenc}     % Input encoding
\usepackage[english,spanish]{babel} % For Spanish texts
\usepackage{FiraSans}           % Use a better font

% Extra packages
\usepackage{graphicx}           % For graphics
\usepackage{verbatim}           % For non-parsed text blocks
\usepackage{microtype}          % Improve PDF text lines
\usepackage{indentfirst}        % Indent first paragraph too
\usepackage{hyperref}           % For intenal and external links
\usepackage[hypcap]{caption}    % For captions
\usepackage{ctable}             % For tables
\usepackage{float}              % For table positions
\usepackage{subcaption}         % For subfigures
\usepackage{fancyhdr}           % For headers and footers
\usepackage{xcolor}             % Colors everywhere
\usepackage{titlesec}           % For customize the title format

% Document properties
\makeatletter
\author{Benito Palacios Sánchez}     \let\Author\@author
\title{Junta Directiva}        \let\Title\@title
\date{19 de diciembre de 2015}         \let\Date\@date
\makeatother

\hypersetup{
    pdfauthor = {\Author},
    pdftitle  = {\Title},
    pdfsubject  = {},
    pdfkeywords = {meeting, ieee},
    pdfcreator  = {TeXLive distro and Atom text editor},
    pdfproducer = {pdflatex}
}

% Define custom colors
\definecolor{BlueTitle}{HTML}{2E5395}

% Define the style for header and footer
\newcommand{\fncymain}{%
    \setlength{\headheight}{85pt}
    \fancyhead[RO,LO,RE,LE]{}
    \fancyhead[C]{
        \includegraphics[height=0.10\textheight,keepaspectratio]
            {../logo_header.png}}
    \renewcommand{\headrulewidth}{0pt}

    \fancyfoot[C]{
        \centering
        \textbf{IEEE Student Branch of Granada} \\
        Campus Fuentenueva. Facultad de Ciencias. Severo Ochoa, S/N. \\
        Sala de alumnos. 18001, Granada, España. \\
        E-mail: \href{mailto:ieeegranada@ieee.org}{ieeegranada@ieee.org}  \\
        Web: \url{http://ieee-urg.org}
    }
}

% Create the document title
\renewcommand{\maketitle}{
    {
        \Huge
        \color{BlueTitle}
        \centering
        \bfseries
        \scshape
        \hrulefill \\
        \Title \\
        \hrulefill
    }
}

% Set the section format
\titleformat{\section}[display]{
    \LARGE
    \color{BlueTitle}
    \filcenter
    \bfseries
    \scshape
    }{}{0pt}{}[{\titlerule[1pt]}]

% Set only one number for subsections
\renewcommand{\thesubsection}{\arabic{subsection}}

% Start!
\begin{document}
    % Page style
    \pagestyle{fancy}
    \fncymain{}

    % Add the title
    \maketitle

    % Date, participantes and location info.
    En Granada, siendo las 10:24 horas del \Date~se encuentran reunidos:
    \begin{itemize}
        \item Benito Palacios Sánchez
        \item Sulaimane Mezzouji
        \item Julian Fernandez Ortiz
        \item Laura Hinojosa Vigo
    \end{itemize}


    \section{Orden del día}
    \begin{enumerate}
        \item Organización de la rama.
    \end{enumerate}


    \section{Desarrollo de la sesión}
    \subsection{Organizacion de la rama}
    Iniciamos la sesión discutiendo posibles mejoras de organización de la rama. Se usará el grupo de Trello de la junta directiva. En la lista de siguiente junta directiva se propondrán los orden del día. En las listas de TODO las tareas.

    Propuestas de mejoras de organización:
    \begin{itemize}
        \item Utilizar el formulario de proyectos para proponer ideas. Al mismo tiempo se expondrá la idea en Telegram y se debatirá y votará mediante democracia real electrónica. Se aprueba verbalmente. En Trello se finalizará la propouesta detallando el proceso. El encargado creará una lista, añadiendo la persona responsable y fecha al final.

        \item Debe de haber reuniones semanales para discutir el estatus.

        \item En el momemnto de organizar una actividad con otra asociación, se debe reunir el presidente. Además, la junta directiva podrá designar un apoderado para hablar en representación.
    \end{itemize}

    Respecto a la organización de los documentos, estos se subirán al Google Drive. Opcionalmente los públicos se podrán publicar en Github.

    \clearpage
    Siendo las 11:05 horas del mismo día se da por terminada la sesión.

    A \Date.
    \vspace{10mm}
    \ctable[pos = h, width = 140 mm]{XX}{}{
        El Secretario & VB del Presidente \NN
                      &                   \NN
                      &                   \NN
                      &                   \NN
                      &                   \NN
                      &                   \NN
                      &                   \NN
        Fdo.: \Author & Fdo.: Sulaimane Mezzouji
    }
\end{document}
